\chapter{HOL}

\section{Syntax}

We can quantify over predicate position.
The same intro and elim rules hold for 2nd-order quantifiers.

These are formally easy to define but hard to express in English!
Some have thought that this richer language is very useful in mathematics and metaphysics.

\begin{exercise}
Could we quantify over operators or quantifiers?
\end{exercise}

There's also 2nd-order identity,
which is governed by SI and LL (intro and elim).

\section{Lambdas}

A predicate says something about such an object. Informally, if you take a
sentence and remove name, leaving a gap, you get a predicate. A predicate isn't
true or false -- as it contains a gap. Rather it may be true \textit{of} some
objects and false \textit{of} others.

% ** Variables as markers

[Following Bostock here; later intro lambdas!]

Consider the sentence 'Horace loves Horace'. This could be understood as applying a one-place predicate to Horace, or ....

To mark this difference, we need different gap markers: '-- loves --' vs '--
loves \_\_'. Instead of gap markers, we can use variables. Thus we can distinguish '$x$ loves $x$' from '$x$ loves $y$'.

Hmm. But now sentence formation is substitution and $F$ isn't a predicate? Well, yes, $F$ is a predicate letter! Move to lambda chapter. We need complex predicates just as we need complex sentences.

\begin{exercise}
Can you think of a way to construct complex singular terms from formulas? [iota]
\end{exercise}


Goldfarb sec. 5.4.

\section{models}

% ** Higher-order models

We can quantify over predicate position.
As a first shot,
we could read ∀XXa as Fa∧Ga∧...
But what if some properties are not expressed by predicates?
I assume that "property" is the label for the kinds of things expressed by predicates.
These plausibly aren't extensions.
Remember that we use extensions to \textit{model} interpretations.
Given that sets of individuals model properties,
it's natural to model second-level predicates as ranging over sets of individuals.

Note that this makes properties "abundant".
We've put no constraints on what predicates can mean.
For any set of individuals,
[F] could be assigned that set.
We could put restrictions on the eligible sets,
but let's not do this.

2nd-order logic quantifies into the position of predicates and propositions. We assume that there are "entities" of the higher type.

We \textit{model} this by assuming that the 2nd-order quantifiers range over arbitrary sets of individuals, just as we modeled the meaning of 2nd-order predicates as arbitrary sets.


\section{Categoricity}



% ** Categoricity

\begin{definition}{Categoricity}{categoricity}
\emph{Categoricity} is the property of a theory that all its models are isomorphic. Explain what this means. For two models $\langle\mathbb{N};0,S,+,\cdot\rangle$ there's a bijection f...
\end{definition}

Second-order PA is categorical.

\begin{exercise}
Explain why a theory without = can never be categorical. [You can always duplicate the elements of a model.]
\end{exercise}

\begin{exercise}
What if you add to PA2 infinitely many statements $a\not=0, a\not=s(0),....$? Is this theory consistent? What do its models look like?
\end{exercise}

Yes, consistent. No models. https://math.stackexchange.com/questions/1005283/whats-an-example-of-a-theory-thats-consistent-yet-has-no-model

% ** Structuralism

The "structuralist" approach to mathematics assumes that the study of arithmetic is the study not of particular things -- numbers -- but of a more abstract structure. The structure can be instantiated in many ways. This structure can only be adequately described in second-order logic.

The same is true for all infinite mathematical structures.

% ** A set theory worry

Remember that standard 2nd order logic assumes that the second-order quantifiers range over sets. One might worry that this undercuts the structuralist approach. We'd now need to be assured that the set-theoretic universe is definite.

Hamlin p.32
