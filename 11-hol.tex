\chapter{HOL}


Note that it is not obvious whether every first-order model represents a genuinely possible interpretation and scenario. There are lots of infinite sets of numbers, for example, many of which are not specifiable in English in any finite way. Could each of them possibly be the meaning of a predicate? We could restrict the possible interpretations of predicates to some subsets of the domain. But to which? Anyway, it makes no difference to the logic. We'll return to this in the case of 2nd-order logic though. (There, if we leave open the higher-order domain, only saying that it's some subset of the powerset of the domain, the class of models becomes much larger. As a result, the logic becomes much weaker.)



Power of SOL for arithmetic:

Single induction axiom

Also can define + and *:

A problem with recursive definitions is that they are not explicit definitions. The two principles implicitly define addition uniquely, but they don't give us an object-language formula φ(x,y,z) that says that x+y=z.

In second-order logic, we can offer such a formula: x+y=z can be defined as $\forall f[(f(0)=x \land \forall v(f(s(v)) = s(f(v)))) \rightarrow f(y)=z]$.

It turns out that there's no way to do this in first-order logic. So we add the two principles as axioms.

https://math.stackexchange.com/questions/449146/why-are-addition-and-multiplication-included-in-the-signature-of-first-order-pea


\section{Syntax}

We can quantify over predicate position.
The same intro and elim rules hold for 2nd-order quantifiers.

These are formally easy to define but hard to express in English!
Some have thought that this richer language is very useful in mathematics and metaphysics.

\begin{exercise}
Could we quantify over operators or quantifiers?
\end{exercise}

There's also 2nd-order identity,
which is governed by SI and LL (intro and elim).

\section{Lambdas}

A predicate says something about such an object. Informally, if you take a
sentence and remove name, leaving a gap, you get a predicate. A predicate isn't
true or false -- as it contains a gap. Rather it may be true \textit{of} some
objects and false \textit{of} others.

% ** Variables as markers

[Following Bostock here; later intro lambdas!]

Consider the sentence 'Horace loves Horace'. This could be understood as applying a one-place predicate to Horace, or ....

To mark this difference, we need different gap markers: '-- loves --' vs '--
loves \_\_'. Instead of gap markers, we can use variables. Thus we can distinguish '$x$ loves $x$' from '$x$ loves $y$'.

Hmm. But now sentence formation is substitution and $F$ isn't a predicate? Well, yes, $F$ is a predicate letter! Move to lambda chapter. We need complex predicates just as we need complex sentences.

\begin{exercise}
Can you think of a way to construct complex singular terms from formulas? [iota]
\end{exercise}


Goldfarb sec. 5.4.

\section{models}

% ** Higher-order models

We can quantify over predicate position.
As a first shot,
we could read ∀XXa as Fa∧Ga∧...
But what if some properties are not expressed by predicates?
I assume that "property" is the label for the kinds of things expressed by predicates.
These plausibly aren't extensions.
Remember that we use extensions to \textit{model} interpretations.
Given that sets of individuals model properties,
it's natural to model second-level predicates as ranging over sets of individuals.

Note that this makes properties "abundant".
We've put no constraints on what predicates can mean.
For any set of individuals,
[F] could be assigned that set.
We could put restrictions on the eligible sets,
but let's not do this.

2nd-order logic quantifies into the position of predicates and propositions. We assume that there are "entities" of the higher type.

We \textit{model} this by assuming that the 2nd-order quantifiers range over arbitrary sets of individuals, just as we modeled the meaning of 2nd-order predicates as arbitrary sets.


\section{Categoricity}



% ** Categoricity

\begin{definition}{Categoricity}{categoricity}
\emph{Categoricity} is the property of a theory that all its models are isomorphic. Explain what this means. For two models $\langle\mathbb{N};0,S,+,\cdot\rangle$ there's a bijection f...
\end{definition}

Second-order PA is categorical.

\begin{exercise}
Explain why a theory without = can never be categorical. [You can always duplicate the elements of a model.]
\end{exercise}

\begin{exercise}
What if you add to PA2 infinitely many statements $a\not=0, a\not=s(0),....$? Is this theory consistent? What do its models look like?
\end{exercise}

Yes, consistent. No models. https://math.stackexchange.com/questions/1005283/whats-an-example-of-a-theory-thats-consistent-yet-has-no-model

% ** Structuralism

The "structuralist" approach to mathematics assumes that the study of arithmetic is the study not of particular things -- numbers -- but of a more abstract structure. The structure can be instantiated in many ways. This structure can only be adequately described in second-order logic.

The same is true for all infinite mathematical structures.

% ** A set theory worry

Remember that standard 2nd order logic assumes that the second-order quantifiers range over sets. One might worry that this undercuts the structuralist approach. We'd now need to be assured that the set-theoretic universe is definite.

Hamlin p.32



Can we settle the continuum hypothesis in second-order ZFC? It isn't derivable.
But if we use a particular set theory to study models of second-order ZFC, then
all of them will decide the CH within ZFC2, but how they decide them depends on
the ambient set theory!

For set theory, categoricity is a delicate matter anyway. Categoricity means that all models are isomorphic, but models are set-theoretic objects, with sets as their domains. In a sense, every model of set theory (first or second order) is a non-standard model!



%%% Local Variables:
%%% mode: latex
%%% TeX-master: "logic3.tex"
%%% End:
