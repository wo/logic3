\begin{tikzpicture}[node distance=30mm, thick]

\iffalse

This starts on the leftmost stroke of the block that we want to shift.
There might be stuff on the left, which we shouldn't touch:
0011101*11  =>  00111010011

q1,1:B,R,q2  Erase the current stroke and move right.
% round 1: 00111010*1
% round 2: 001110100*01
q2,B:1,L,q7  If there's a blank here, print a stroke; we're done.
% round 2: 00111010*011
q2,1:1,R,q3  If there's a stroke here, move right.
% round 1: 001110101*00
q3,1:1,R,q3  While you're on a stroke, keep moving right.
q3,B:B,R,q4  Move right again past the blank.
% 0011101010*0
q4,1:1,R,q4  While you're on a stroke, keep moving right.
q4,B:1,L,q5  Print a stroke at the first blank and go left.
% 001110101*01
q5,1:1,L,q5  While you're on a stroke, move left.
q5,B:B,L,q6  Go left past the first blank.
% 00111010*101
q6,1:1,L,q6  While you're on a stroke, move left.
% 0011101*0101
q6,B:B,R,q1  Move right to the first stroke and start again.
% 00111010*101

NB: This machine ends on the blank just before the moved block.

JSON format for https://math.hws.edu/eck/js/turing-machine/TM.html:
(state no, read symbol, write symbol, next state no, direction)

"rules": [
      [ 0, "0", "0", 0, "R" ],
      [ 0, "1", "0", 1, "R" ],
      [ 1, "0", "1", 7, "L" ],
      [ 1, "1", "1", 2, "R" ],
      [ 2, "0", "0", 3, "R" ],
      [ 2, "1", "1", 2, "R" ],
      [ 3, "1", "1", 3, "R" ],
      [ 3, "0", "1", 4, "L" ],
      [ 4, "1", "1", 4, "L" ],
      [ 4, "0", "0", 5, "L" ],
      [ 5, "1", "1", 5, "L" ],
      [ 5, "0", "0", 0, "R" ]
]
\fi

%--- states -------------------------------------------------------------------
\node[state] (q0) {$q_0$};
\node[state,below of=q0]  (q1) {$q_1$};
\node[state,left of=q1]   (q7) {$q_{7}$};
\node[state,right of=q1]   (q2) {$q_2$};
\node[state,right of=q2]   (q3) {$q_3$};
\node[state,above of=q3]   (q4) {$q_4$};
\node[state,left of=q4]   (q5) {$q_5$};

%--- transitions --------------------------------------------------------------

\path[->] (q0) edge[loop above] node {B: BR} (q0);
% [ 0, "0", "0", 0, "R" ],
\path[->] (q0) edge node[left] {1: BR} (q1);
% [ 0, "1", "0", 1, "R" ],

\path[->] (q1) edge node {B: 1L} (q7);
% [ 1, "0", "1", 7, "L" ],
\path[->] (q1) edge node {1: 1R} (q2);
% [ 1, "1", "1", 2, "R" ],

\path[->] (q2) edge[loop above] node {1: 1R} ();
\path[->] (q2) edge node {B: BR} (q3);
% [ 2, "0", "0", 3, "R" ],

\path[->] (q3) edge[loop right] node {1: 1R} ();
% [ 3, "1", "1", 3, "R" ],
\path[->] (q3) edge node[right] {B: 1L} (q4);
% [ 3, "0", "1", 4, "L" ],

\path[->] (q4) edge[loop right] node {1: 1L} ();
% [ 4, "1", "1", 4, "L" ],
\path[->] (q4) edge node[above] {B: BL} (q5);
% [ 4, "0", "0", 5, "L" ],

\path[->] (q5) edge[loop above] node {1: 1L} ();
\path[->] (q5) edge node[above] {B: BR} (q0);




\end{tikzpicture}
