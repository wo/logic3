% --------------------------------------------------------------
% State-transition diagram for the example Turing machine
% --------------------------------------------------------------
\begin{tikzpicture}[>=stealth', shorten >=1pt, auto,
                    node distance=5cm, thick]
  % ------------------------------------------------------------
  % Nodes
  % ------------------------------------------------------------
  \node[state]                 (q0) {$q_{0}$};
  \node[state, right of=q0]    (q1) {$q_{1}$};
  \node[state, right of=q1]    (q2) {$q_{2}$};

  % ------------------------------------------------------------
  % Edges  (label =  read : write , move )
  % ------------------------------------------------------------
  % --- from q0 to q1 ------------------------------------------
  \path[->] (q0) edge[bend left=20]  node[above] {$1:BR$} (q1);  % read 1
  \path[->] (q0) edge[bend right=20] node[below] {$B:1L$} (q1);  % read B

  % --- from q1 to q2 ------------------------------------------
  \path[->] (q1) edge[bend left=20]  node[above] {$1:BL$} (q2);  % read 1
  \path[->] (q1) edge[bend right=20] node[below] {$B:BR$} (q2);  % read B
\end{tikzpicture}
% --------------------------------------------------------------
