\setcounter{chapter}{5}
\chapter{Turing computability}\label{ch:turing}

% Follow IGT ch.41?

In 1936,
Alan Turing introduced a formal model of computation
by defining a simple type of computer -- now known as a \emph{Turing machine} -- that,
he suggested,
can implement any mechanical algorithm.

\section{Turing machines}\label{sec:tm}

Let's think about what is involved in following an algorithm.
An algorithm converts some input string into an output string.
Let's assume that
the input string is received on a piece of paper.
The algorithm specifies what one should do with that string,
providing step-by-step instructions to
add, remove, or change symbols on the paper,
until the output string is produced.
At each step in the process,
the algorithm clearly specifies what to do next,
based on what's currently on the paper
and on the current stage of the computation.
When the computation is finished,
the output string must be marked as such on the paper,
perhaps by circling or underlining it.
For definiteness,
let's stipulate that
the algorithm should contain instructions to erase everything else on the paper,
leaving only the output.

A Turing machine is a machine that implements a process of this kind.
It reads and writes symbols on a piece of paper,
thereby converting an input string into an output string
by following precise, step-by-step instructions.

Turing's key insight was that
such a machine can be designed in a very simple way.
To begin,
we can assume that
the paper on which the machine operates is a single strip of paper:
any algorithm that requires writing symbols above or below other symbols
can be reformulated as an algorithm that only requires writing symbols
to the left or right of other symbols.
A Turing machine therefore operates on a \emph{tape}
in which the symbols are always arranged in a single line.
The tape is divided into ``cells'' or ``squares'',
each of which can hold a single symbol.
Since we're interested in what can be computed in principle,
without worrying about practical limitations,
we assume that the tape is unbounded.
(If the machine reaches the end of the tape,
the tape is automatically extended.)

Next,
we make the steps in the computation as simple as possible.
Evidently,
any instruction for writing down a sequence of symbols
can be broken down into a sequence of instructions
for writing down a single symbol.
We'll therefore assume that
at each step,
a Turing machine writes at most one symbol onto its tape.
Similarly,
we assume that a Turing machine
can only read a single cell on its tape at a time.
Any instruction for reading larger chunks of the tape
can be broken down into instructions for
reading single cells.

At each step,
a Turing machine therefore operates on a single cell of its tape.
We say that it has a \emph{head} that is positioned on this cell.
At each step,
the machine can read the content of the current cell;
it can erase that content or replace it with a different symbol,
and it can move its head left or right,
by one cell.

For definiteness,
we assume that
each step involves all these actions.
That is,
each step in a Turing machine computation consists of three parts:

\begin{cenumerate}
  \item Read the content of the current cell;
  \item Erase the content of the current cell and
  either leave it blank or write a new symbol onto it;
  \item Move the head one cell to the left or right.
\end{cenumerate}

We can make one last simplification.
Every known algorithm operates on strings from a finite alphabet.
We can code these strings as numbers greater than 0.
Each such number $n$ can be written in \emph{unary} notation,
as a sequence of $n$ \emph{strokes}.
Since there are effective algorithms for converting
the original strings into sequences of strokes and back,
any algorithm that operates on the original strings can be converted
into an algorithm that operates on sequences of strokes.
We'll therefore assume that
the only symbol available to a Turing machine is the stroke.

In sum,
a Turing machine has an unbounded tape,
divided into cells,
each of which can hold either a stroke or be blank.
The machine works in steps,
in accordance with a predefined program.
At each step,
the machine's head is positioned at a particular cell of the tape.
A step involves
reading the content of that cell,
replacing it with either a blank or a stroke,
and moving the head one cell to the left or right.

A program for a Turing machine specifies what the machine does at each step.
This generally depends on the content of the current cell.
A simple program might look as follows:

\begin{enumerate}[itemsep=2pt,parsep=0pt,topsep=6pt,partopsep=2pt, labelwidth=4em,leftmargin=4.5em]
  \item[\emph{Step 1}.] If the current cell contains a stroke, erase it and move right;
        if the current cell is blank, write a stroke and move left.
  \item[\emph{Step 2}.] If the current cell contains a stroke, leave it and move left;
        if the current cell is blank, write a stroke and move left.
\end{enumerate}

We're not assuming that
a machine that executes this program
would somehow read and understand the instructions.
Turing machines only read strokes or blanks on their tape.
A machine that executes the above program
would simply be wired to follow the two instructions,
one after the other,
and then stop.

To build such a machine,
we would need an internal switch or counter
to keep track of where it is in the computation --
whether it should follow instruction 1 or instruction 2.
The machine might,
for example,
have a switch that can be in one of two positions,
``up'' and ``down''.
We could then build it in such a way that
it follows instruction 1 if the switch is up
and instruction 2 if the switch is down.
The switch would start in the up position and flip to down after
the machine has finished following the first instruction.

To allow for programs with more than two instructions,
we must allow the switch to have any finite number of positions.
These switch positions are called \textit{states} of the machine,
and labelled $q_{0}, q_1, q_2, \ldots$.
It doesn't matter how they are implemented.
You can think of each state as indicating a ``line'' in
the program the machine is executing.

\begin{exercise}
  What does the above machine do if it starts
  (a) on a tape with a single stroke under its head?
  (b) on an empty tape?
\end{exercise}

Many algorithms involve repeating certain steps.
The algorithms you've learned for written addition and multiplication,
for instance,
probably go through each digit in the decimal representation of the input numbers,
asking you to perform the same operations for each digit.
To implement such an algorithm,
a Turing machine must be able to go into the same state more than once.
Here is a program for a machine of this kind.

\begin{enumerate}[itemsep=2pt,parsep=0pt,topsep=6pt,partopsep=2pt, labelwidth=6em,leftmargin=6.5em]
  \item[\emph{Instruction 1}.] If the current cell contains a stroke, erase it and move right, then process Instruction 2;
        if the current cell is blank, write a stroke, move right, and halt.
  \item[\emph{Instruction 2}.] If the current cell contains a stroke, erase it and move right;
        if the current cell is blank, write a stroke and move right;
        either way, continue with Instruction 1.
\end{enumerate}

A machine that implements this program
still needs two states,
$q_{0}$ and $q_{1}$.
In state $q_{0}$,
it follows instruction 1;
in state $q_{1}$,
it follows instruction 2.
Each instruction effectively specifies three actions:
\begin{itemize*}
  \item what to write into the current cell (a stroke or a blank);
  \item whether to move left or right;
  \item what state to go into next.
\end{itemize*}

We can introduce a compact notation for these instructions,
using, for example,
`$1,\mathrm{R},q_1$' to mean that
the machine should replace the content of the current cell by a stroke,
move right,
and go into state $q_{1}$,
and `$\mathrm{B},\mathrm{L},q_0$' for
``empty the current cell, move left, and go into state $q_0$''.
The above program can then be written as a table:

\begin{center}
  \begin{tabular}{c | c c}
    \cellcolor{gray!20} & \cellcolor{gray!20} 1 & \cellcolor{gray!20} B \\
    \hline
    \cellcolor{gray!20} $q_0$ & $\mathrm{B}, \mathrm{R}, q_1$ & $1, \mathrm{R}, q_2$ \\
    \cellcolor{gray!20} $q_1$ & $\mathrm{B}, \mathrm{R}, q_0$ & $1, \mathrm{R}, q_0$ \\
    \cellcolor{gray!20} $q_2$ &  & \\
  \end{tabular}
\end{center}

\noindent%
Each cell in the table holds the instruction for what to do
in a given state (the row) when reading a given symbol (the column).
I've added a third state, $q_{2}$,
so that the directive to halt can be represented as the directive
to go into a new state
for which there are no instructions.

There are other ways to represent the same program.
We could, for example, package it into a list of quintuples:
\begin{equation*}
  (q_{0}, 1, \mathrm{B}, \mathrm{R}, q_{1}), \quad
  (q_{0}, \mathrm{B}, 1, \mathrm{R}, q_{2}), \quad
  (q_{1}, 1, \mathrm{B}, \mathrm{R}, q_{0}), \quad
  (q_{1}, \mathrm{B}, 1, \mathrm{R}, q_{0}).
\end{equation*}

\noindent%
Here,
$\t{q_{0}, 1, \mathrm{B}, \mathrm{R}, q_{1}}$ means that
if the machine is in state $q_{0}$ and reads a stroke,
then it should erase the stroke (``write a blank''), move right, and go into state $q_{1}$.
Similarly for the other entries in the list.

Another popular way to represent Turing machine programs is as a flow chart.
Here is the same machine again:

\begin{center}
  % --------------------------------------------------------------
% State-transition diagram:  unary machine
%   even block  -> halts
%   odd  block  -> infinite loop
% --------------------------------------------------------------
\begin{tikzpicture}[node distance=3cm, thick]

  %-------------------------------------------------------------
  % States
  %-------------------------------------------------------------
  \node[state]                 (q0) {$q_{0}$};
  \node[state, right of=q0]             (q1) {$q_{1}$};
  \node[state, below=2.6cm of q0] (q2) {$q_{2}$};

  %-------------------------------------------------------------
  % Transitions on symbol 1  (the working loop)
  %-------------------------------------------------------------
  \path[->] (q0) edge[bend left=15]  node[above] {1: BR} (q1);
  \path[->] (q1) edge[bend left=20]  node[below] {1: BR} (q0);
  \path[->] (q1) edge[bend left=60]  node[below] {B: 1R} (q0);

  %-------------------------------------------------------------
  % Blank seen in q0  ->  halt
  %-------------------------------------------------------------
  \path[->] (q0) edge[bend right=20]  node[left] {B: 1R} (q2);

  %-------------------------------------------------------------
  % Blank seen in q1  ->  loop for ever
  %-------------------------------------------------------------
  % \path[->] (q1) edge[loop above]    node {B: BR} ();

\end{tikzpicture}
% --------------------------------------------------------------

\end{center}

The nodes in the chart represent the states.
Each arrow represents an instruction.
`$1: \mathrm{B}\mathrm{R}$' means:
`if you read 1, write a blank, and move right'.
The new state is given by the node to which the arrow points.

\begin{exercise}\label{ex:first-tm}
  Can you figure out what this machine does if it starts
  at the left end of a sequence of strokes
  on an otherwise empty tape?
\end{exercise}

Let's do this exercise together.
The machine starts in state $q_{0}$, reading the first stroke.
It erases the stroke and moves right,
entering state $q_{1}$.
It reads the second stroke,
erases it, moves right, and goes back into state $q_{0}$.
It keeps alternating between $q_{0}$ and $q_{1}$ in this way,
moving right and erasing strokes,
until it reaches the first blank.
At that point,
the machine is either in state $q_{0}$ or $q_{1}$,
depending on whether the original tape had an even or an odd number of strokes.
If the number of strokes was even,
the machine is now in state $q_{0}$;
it reads the blank, prints a stroke, moves right and halts (in state $q_{2}$).
If the tape originally had an odd number of strokes,
the machine is in state $q_{1}$ when it reaches the first blank.
It prints a stroke, moves right, and goes into $q_{0}$.
It then reads another blank, prints another stroke, moves right, and halts.

The machine implements an algorithm for
deciding whether the input sequence has an even or odd number of strokes.
If even, the output is a single stroke;
if odd, the output is two strokes.

% Note that the loop between states $q_{0}$ and $q_{1}$ is \emph{unbounded}.
% -- Ah, not true.
% The input on the tape fixes the number of iterations,
% and that makes the loop a for loop...
% The program doesn't specify in advance how many times to go through the loop.
% The loop ends if and when the machine reads the first blank.
% Unbounded loops are also called \emph{while loops},
% in contrast to \emph{for loops},
% which run a predefined number of times.
% The labels reflect the fact that in
% many programming languages,
% one writes something like
%
% \begin{verbatim}
%   while (condition) { instruction }
% \end{verbatim}
%
% to indicate that the instruction should be repeated as long as the condition is true,
% whereas one typically writes something like the following if
% the number of repetitions is known in advance:
%
% \begin{verbatim}
%   for (i = 0; i < n; i++) { instruction }
% \end{verbatim}
%
% (Technically,
% in just about every language that has for loops,
% one can also write a for loop that runs indefinitely,
% and one can write while loops that run a predefined number of times;
% so the labels have to be taken with a grain of salt.)

\begin{exercise}
  In computer programming,
  it is important to check for edge cases.
  Does the program correctly classify the empty input as having an even number of strokes?
\end{exercise}

\begin{exercise}\label{ex:tm-add-stroke}
  Design a Turing machine that extends any input sequence of strokes
  by one stroke:
  when starting on the left-most stroke of a sequence of $n$ strokes
  on an otherwise blank tape,
  the machine eventually halts on an otherwise blank tape with
  a sequence of $n + 1$ strokes.
\end{exercise}

\begin{exercise}\label{ex:bb3}
  Draw a flow chart for the machine given by the following quintuples:
  $(q_0, \mathrm{B}, 1, \mathrm{R}, q_1)$,
  $(q_0, 1, 1, \mathrm{L}, q_2)$,
  $(q_1, \mathrm{B}, 1, \mathrm{L}, q_{0})$,
  $(q_1, 1, 1, \mathrm{R}, q_{1})$,
  $(q_2, \mathrm{B}, 1, \mathrm{L}, q_{1})$.
  Can you figure out what this machine does,
  when started on a blank tape?
  % It writes six strokes, then halts.
%
% 3-state BB from https://en.wikipedia.org/wiki/Turing_machine_examples
%                  ┌────────────── Current state ──────────────┐
% Tape symbol      │     q₀              q₁              q₂     │
% ─────────────────┼────────────────────────────────────────────┤
% B   (blank)      │  1 , R ,  q₁   1 , L ,  q₀    1 , L ,  q₁ │
% 1   (stroke)     │  1 , L ,  q₂   1 , R ,  q₁    1 , N , HALT │
%                  └────────────────────────────────────────────┘
\end{exercise}


\section{Computing arithmetical functions}\label{sec:tm-arith}

A Turing machine converts a pattern of strokes and blanks on its tape into another pattern of strokes and blanks.
To compute a function that
doesn't take such patterns as input or output,
we must code the inputs and outputs as patterns of strokes and blanks.

Let's look at functions on the natural numbers.
In the previous section,
I suggested that we could code each number $n$ as a sequence of $n$ strokes.
This works,
but it has the downside that
we can't distinguish between empty cells and cells that store the number 0.
In this section,
I'll therefore use a slightly different coding scheme,
in which
each number $n$ is represented by a sequence of $n+1$ strokes:
the number 0 is coded by a single stroke,
1 by a sequence of two strokes,
and so on.

We say that a Turing machine
\emph{computes a function $f$ from $\mathbb{N}$ to $\mathbb{N}$}
if,
whenever it starts on the left-most stroke of a sequence of $n+1$ strokes on an otherwise blank tape,
it eventually halts on a tape with a sequence of $f(n) + 1$ strokes on an otherwise blank tape.
A function $f$ from $\mathbb{N}$ to $\mathbb{N}$ is \emph{Turing-computable}
if it is computed by some Turing machine.

These definitions can obviously be generalized to functions with more than one argument.
If a function takes $n$ numbers as input,
we stipulate that
these numbers must be represented by $n$ blocks of strokes,
separated by a blank.
For example,
if we want a Turing machine to add the numbers 2 and 3,
we would start it on a tape that looks like this:
\[
  \ldots \turingtape{{ , ,1,1,1, ,1,1,1,1, , }}{2} \ldots
\]

\begin{exercise}
  Since blanks and strokes effectively give us two symbols,
  one might suggest that we could code numbers in binary,
  so that 0 is coded as a blank ($\mathrm{B}$),
  1 as a stroke ($1$),
  2 as $1\mathrm{B}$,
  3 as $11$,
  4 as $1\mathrm{B}\mathrm{B}$,
  and so on.
  Explain why this doesn't work.
\end{exercise}

In exercise \ref{ex:tm-add-stroke},
you designed a Turing machine that adds a single stroke to
the sequence of strokes at which it starts.
By our present conventions,
this machine computes the successor function
that takes a number as input and returns that number plus 1.

Here is a machine that computes the ``times 2'' function:
it converts a sequence of $n+1$ strokes into a sequence of $2n+1$ strokes.

\begin{center}
  % --------------------------------------------------------------
% State-transition diagram: Doubles the number of strokes
%   Input:  n strokes (unary)
%   Output: 2n-1 strokes (unary)
%   Strategy: Erase each stroke from left, add two on right
%   This is essentially taken from the open logic project,
%   but I added the -1 at the end.
% --------------------------------------------------------------
\begin{tikzpicture}[node distance=3.5cm, thick]
  %-------------------------------------------------------------
  % States
  %-------------------------------------------------------------
  \node[state]           (q0) {$q_{0}$};
  \node[state, right of=q0]       (q1) {$q_{1}$};
  \node[state, right of=q1]       (q2) {$q_{2}$};
  \node[state, below=2cm of q2] (q3) {$q_{3}$};
  \node[state, left of=q3]        (q4) {$q_{4}$};
  \node[state, left of=q4]        (q5) {$q_{5}$};
  \node[state, left of=q0]        (q6) {$q_{6}$};

  %-------------------------------------------------------------
  % q0: Erase leftmost stroke
  %-------------------------------------------------------------
  \path[->] (q0) edge              node[above] {1: BR} (q1);

  %-------------------------------------------------------------
  % q1: Skip remaining original strokes to find gap
  %-------------------------------------------------------------
  \path[->] (q1) edge[loop above]  node {1: 1R} ();
  \path[->] (q1) edge              node[above] {B: BR} (q2);

  %-------------------------------------------------------------
  % q2: Skip new strokes to find end
  %-------------------------------------------------------------
  \path[->] (q2) edge[loop above]  node {1: 1R} ();
  \path[->] (q2) edge              node[right] {B: 1R} (q3);

  %-------------------------------------------------------------
  % q3: Add second stroke, then start moving back
  %-------------------------------------------------------------
  \path[->] (q3) edge[loop below]  node {B: 1L} ();
  \path[->] (q3) edge              node[below] {1: 1L} (q4);

  %-------------------------------------------------------------
  % q4: Move left through new strokes
  %-------------------------------------------------------------
  \path[->] (q4) edge[loop below]  node {1: 1L} ();
  \path[->] (q4) edge              node[below] {B: BL} (q5);

  %-------------------------------------------------------------
  % q5: Move left through blanks (erased area)
  %-------------------------------------------------------------
  \path[->] (q5) edge[loop below]  node {1: 1L} ();
  \path[->] (q5) edge              node[left]  {B: BR} (q0);

  %-------------------------------------------------------------
  % q5: Add final -1
  %-------------------------------------------------------------
  \path[->] (q0) edge node[above] {B: BR} (q6);
  \path[->] (q6) edge[loop above] node {1: BL} ();
\end{tikzpicture}
% --------------------------------------------------------------

\end{center}

The output sequence is constructed to the right of the input,
with a blank as a separator.
The machine successively removes one stroke from the input,
then moves right past the first blank after the input,
then moves right past all strokes that follow the blank,
prints two strokes, and returns to the left-most stroke
on what remains of the input sequence,
until that sequence is completely erased.
At this point,
the machine has created a block of $2n+2$ strokes.
It removes the left-most stroke of this block and halts.

\begin{exercise}
  Design a Turing machine that computes addition:
  when started at the left end of a sequence of $n+1$ strokes
  followed by a blank followed by $m+1$ strokes,
  the machine halts on a tape with $n+m+1$ consecutive strokes.
  % Erase the first stroke, move right to the first blank, replace it by a stroke,
  % then move either to the left or the right end of the sequence and remove
  % the final stroke.
\end{exercise}

To implement more complex algorithms,
it helps to think in terms of subroutines.
Let's tackle multiplication.
A Turing machine that computes multiplication
would start at the left end of a block of $n+1$ strokes,
followed by a blank, followed by another block of $m+1$ strokes,
and eventually halt on a tape with $n \times m + 1$ consecutive strokes.
How could we design such a machine?

We could use the first block of strokes as a counter,
as in the doubling machine:
we'd erase one stroke at a time from the left block;
for each stroke that's erased,
we add $m$ strokes to the second block;
we do this until the counter block has only two strokes left,
at which point we erase these strokes and halt.
(Do you understand why we'd stop when there are two strokes left in the counter?)

But how can we repeatedly add $m$ strokes to the second block?
After $i$ iterations,
the first block would have $n+1-i$ strokes and the second $i\times m + 1$.
It's hard to extract from this the original value $m$.
So here's a better idea:
instead of directly adding $m$ strokes to the second block,
we insert $m$ \emph{blanks} \emph{inside} the second block,
after its first stroke.
That is,
we shift the last $m$ strokes of the second block $m$ squares to the right.
When we're done,
we fill all these blanks with strokes.

For example, consider the case of $n=3$ and $m=2$.
The input tape is
\[
  \ldots \turingtape{{1,1,1,1, ,1,1,1, , , , }}{0} \ldots
\]
The desired output is a sequence of $3 \times 2 +1 = 7$ strokes.
We begin by erasing the first stroke in the left block (the counter block).
We now have
\[
  \ldots \turingtape{{ ,1,1,1, ,1,1,1, , , , }}{0} \ldots
\]
Then we shift the two last strokes in the right block two squares to the right:
\[
  \ldots \turingtape{{ ,1,1,1, ,1, , ,1,1, , }}{8} \ldots
\]
Then we start over,
erasing another stroke in the counter block
and shifting the two strokes on the right by two more squares:
\begin{gather*}
  \ldots \turingtape{{ , ,1,1, ,1, , ,1,1, , }}{1} \ldots \\
  \ldots \turingtape{{ , ,1,1, ,1, , , , ,1,1}}{10} \ldots
\end{gather*}
Now there are only two strokes left in the counter.
We erase these two strokes
and fill in all the blanks we've inserted in the right block:
\[
  \ldots \turingtape{{ , , , , ,1,1,1,1,1,1,1}}{5} \ldots
\]
We have the desired output of seven strokes.

Most of this is straightforward to implement.
The only slightly tricky part is the subroutine for shifting the strokes in the second block.
Let's think of this as a separate task.
Let's assume that the head is at the first of the $m$ strokes that we want to shift by $m$ squares to the right.
The machine that achieves this
should not change anything to the left of its starting position.
Here is a machine that does the job.
Let's call it `SHIFT BLOCKS'.

\begin{center}
  \begin{tikzpicture}[node distance=30mm, thick]

\iffalse

This starts on the leftmost stroke of the block that we want to shift.
There might be stuff on the left, which we shouldn't touch:
0011101*11  =>  00111010011

q1,1:B,R,q2  Erase the current stroke and move right.
% round 1: 00111010*1
% round 2: 001110100*01
q2,B:1,L,q7  If there's a blank here, print a stroke; we're done.
% round 2: 00111010*011
q2,1:1,R,q3  If there's a stroke here, move right.
% round 1: 001110101*00
q3,1:1,R,q3  While you're on a stroke, keep moving right.
q3,B:B,R,q4  Move right again past the blank.
% 0011101010*0
q4,1:1,R,q4  While you're on a stroke, keep moving right.
q4,B:1,L,q5  Print a stroke at the first blank and go left.
% 001110101*01
q5,1:1,L,q5  While you're on a stroke, move left.
q5,B:B,L,q6  Go left past the first blank.
% 00111010*101
q6,1:1,L,q6  While you're on a stroke, move left.
% 0011101*0101
q6,B:B,R,q1  Move right to the first stroke and start again.
% 00111010*101

NB: This machine ends on the blank just before the moved block.

JSON format for https://math.hws.edu/eck/js/turing-machine/TM.html:
(state no, read symbol, write symbol, next state no, direction)

"rules": [
      [ 0, "0", "0", 0, "R" ],
      [ 0, "1", "0", 1, "R" ],
      [ 1, "0", "1", 7, "L" ],
      [ 1, "1", "1", 2, "R" ],
      [ 2, "0", "0", 3, "R" ],
      [ 2, "1", "1", 2, "R" ],
      [ 3, "1", "1", 3, "R" ],
      [ 3, "0", "1", 4, "L" ],
      [ 4, "1", "1", 4, "L" ],
      [ 4, "0", "0", 5, "L" ],
      [ 5, "1", "1", 5, "L" ],
      [ 5, "0", "0", 0, "R" ]
]
\fi

%--- states -------------------------------------------------------------------
\node[state] (q0) {$q_0$};
\node[state,below of=q0]  (q1) {$q_1$};
\node[state,left of=q1]   (q7) {$q_{7}$};
\node[state,right of=q1]   (q2) {$q_2$};
\node[state,right of=q2]   (q3) {$q_3$};
\node[state,above of=q3]   (q4) {$q_4$};
\node[state,left of=q4]   (q5) {$q_5$};

%--- transitions --------------------------------------------------------------

\path[->] (q0) edge[loop above] node {B: BR} (q0);
% [ 0, "0", "0", 0, "R" ],
\path[->] (q0) edge node[left] {1: BR} (q1);
% [ 0, "1", "0", 1, "R" ],

\path[->] (q1) edge node {B: 1L} (q7);
% [ 1, "0", "1", 7, "L" ],
\path[->] (q1) edge node {1: 1R} (q2);
% [ 1, "1", "1", 2, "R" ],

\path[->] (q2) edge[loop above] node {1: 1R} ();
\path[->] (q2) edge node {B: BR} (q3);
% [ 2, "0", "0", 3, "R" ],

\path[->] (q3) edge[loop right] node {1: 1R} ();
% [ 3, "1", "1", 3, "R" ],
\path[->] (q3) edge node[right] {B: 1L} (q4);
% [ 3, "0", "1", 4, "L" ],

\path[->] (q4) edge[loop right] node {1: 1L} ();
% [ 4, "1", "1", 4, "L" ],
\path[->] (q4) edge node[above] {B: BL} (q5);
% [ 4, "0", "0", 5, "L" ],

\path[->] (q5) edge[loop above] node {1: 1L} ();
\path[->] (q5) edge node[above] {B: BR} (q0);




\end{tikzpicture}

\end{center}

We can now plug this into a multiplication machine,
using the algorithm I've just described:

\begin{center}
  \tikzset{
  subroutine/.style = {rectangle,double,draw,
                       minimum width  = 25mm,
                       minimum height = 11mm,
                       text width     = 20mm,
                       align          = center,
                       font           = \footnotesize},
}


\begin{tikzpicture}[node distance=30mm, thick]

  %-------------------------------------------------------------
  % States
  %-------------------------------------------------------------
  \node[state]           (q0) {$q_{0}$};
  \node[state, right of=q0]       (q1) {$q_{1}$};
  \node[state, right of=q1]       (q2) {$q_{2}$};
  \node[state, right of=q2]       (q8) {$q_{8}$};
  \node[state, below right of=q8]       (q9) {$q_{9}$};
  \node[state, below of=q9]      (q10) {$q_{10}$};
  \node[state, below right of=q2]       (q3) {$q_{3}$};
  \node[state, below of=q3]       (q4) {$q_{4}$};
  \node[subroutine, left of=q4]   (shift) {SHIFT\\BLOCKS};
  \node[state, left of=shift] (q5) {$q_{5}$};
  \node[state, left of=q5]        (q6) {$q_{6}$};
  \node[state, above of=q6]       (q7) {$q_{7}$};

% Erase the current stroke and move right.
\path[->] (q0) edge node {1:BR} (q1);

% Move right once more.
\path[->] (q1) edge node {1:1R} (q2);

% If there's a stroke here, move right.
\path[->] (q2) edge node {1:1R} (q3);

% Keep moving right past all strokes in the counter block.
\path[->] (q3) edge[loop right] node {1:1R} ();

% Move right past the blank after the counter block.
\path[->] (q3) edge node {B:BR} (q4);

% Move right past the initial stroke of the second argument.
% NB: The subroutine moves past any more blanks.
\path[->] (q4) edge node {1:1R} (shift);

% The subroutine ends on the blank just before the moved block.
% Move left from here.
\path[->] (shift) edge node {B:BL} (q5);

% Keep moving left past all blanks in the right block.
\path[->] (q5) edge[loop below] node {B:BL} ();

% Move left past the initial stroke in the right block.
\path[->] (q5) edge node {1:1L} (q6);

% Move left past the separator blank.
\path[->] (q6) edge node {B:BL} (q7);

% Move to the left-most stroke of the counter block.
\path[->] (q7) edge[loop left] node {1:1L} ();
\path[->] (q7) edge node {B:BR} (q0);

% Now for the tidy-up routine.
% This is triggered if we find a blank at q2,
% which happens iff there were only two strokes in the counter block;
% We've erased one, and we're now on the separator block:
% ...1*B1BBBBB11111...
% We need to erase the counter and fill the blanks in the right block.

% Move left.
\path[->] (q2) edge node {B:BL} (q8);

% Erase the last counter stroke.
\path[->] (q8) edge[loop above] node {1:BR} (q8);

% Move right past the separator blank.
\path[->] (q8) edge node {B:BR} (q9);

% Move right past the initial stroke of the right block.
\path[->] (q9) edge node {1:1R} (q10);

% Replace any blanks by strokes.
\path[->] (q10) edge[loop below] node {B:1R} (q10);

\end{tikzpicture}

\end{center}

In $q_{0}$,
this machine removes the current stroke in the counter block.
It then moves right twice.
If it lands on a blank,
there is only one stroke left in the counter block,
and the machine goes into the cleanup routine $q_{8}$--$q_{10}$,
where it erases the remaining counter stroke and
fills the blanks in the right block.
Alternatively,
if there are further strokes in the counter block,
the machine moves right past the counter block,
past the separator blank,
and past the first stroke of the right block.
It then calls the 'SHIFT BLOCKS' subroutine
(which I've conveniently defined so that
if it starts on a blank then it first moves right until it finds a stroke).
After the subroutine,
the machine moves back to the left end of the counter block.

It takes some practice and patience to design Turing machines that compute arithmetical functions.
In the next chapter, we'll show with one very general argument that
a wide range of arithmetical functions can be computed by Turing machines.

\begin{exercise}
  My multiplication machine has a bug:
  it doesn't correctly deal with certain edge cases.
  Can you find the bug?
  Can you fix it?
  % Design an improved machine that embeds the multiplication machine I've defined
  % and correctly deals with the case of 0 as the first argument.
  % Move right once.
  % If we're on a blank, the first argument is zero.
  % The output should also be zero.
  % So move right once again to the first stroke of the second argument.
  % Keep erasing 1s and moving right until you reach a blank.
  % If, after moving right at the start, we're not on a blank,
  % move left again and call my machine.
\end{exercise}

  % Design a Turing machine that computes the function $f(n) = n^2$.
  % (Hint: Use the machine that doubles the input, and run it $n$ times.)

\begin{exercise}\label{ex:max}
  Design a Turing machine that computes the function $\max(x,y)$
  that takes two numbers as input and returns the larger of the two.
\end{exercise}

\section{Universal Turing machines}\label{sec:universal-tm}

% Turing machines are highly simplified models of computation,
% stripped down to the bare essentials.
% They are not meant to be practical.
% The kinds of computers we have today aren't based on the architecture of Turing machines.
% For example,
% the computers we use as laptops or phones have memory registers from which the processor can read (and to which it can write) in one command,
% without painfully traversing along a memory tape.
% This is convenient,
% but it isn't essential.
% Anything you can compute with designated memory registers
% can also be computed by a Turing machine.

Every Turing machine computes a particular function.
There is a machine for computing addition,
another for multiplication,
and so on.
As Turing pointed out,
one can also design a ``universal'' Turing machine that
can compute \emph{any} computable function.
Such a machine takes as input
an algorithm for computing a function
as well as the arguments to that function.
For example,
if we supply the machine with an algorithm for addition and the numbers 2 and 3,
it would compute the output 5.
If we give it an algorithm for multiplication and the numbers 2 and 3,
it would compute the output 6.

There are different ways of representing an algorithm.
A natural choice,
in the present context,
is to use Turing machine specifications.
Our universal Turing machine $U$ will therefore take as input
a specification of a Turing machine $M$,
as well as some input $I$ for $M$.
Its output will be the output produced by $M$ on input $I$.

Let's think about how we could build such a machine.
To begin,
we need to code specifications of Turing machines as patterns of strokes and blanks,
so that we can feed them as input to the universal machine.

We know that every Turing machine can be represented as a list of quintuples of the form
\[
  \t{q_{i}, s, s', d, q_{j}},
\]
where $q_{i}$ and $q_{j}$ are states,
$s$ and $s'$ are tape symbols (stroke or blank),
and $d$ is a direction (left or right).
We can code each of these components by a string of strokes,
using (say) $i+1$ strokes for state $q_{i}$,
one stroke for the blank and for `left',
and two strokes for the stroke and for `right'.
We can then represent a quintuple by putting
its component codes end to end,
separated by a blank.
The quintuple $\t{q_{0}, 1, \mathrm{B}, \mathrm{R}, q_{1}}$, for example,
would be coded by \inlineturingtape{{1, ,1,1, ,1, ,1,1, ,1,1}}{100}.
To represent an entire machine,
we put all the codes for its quintuples end to end,
separated by (say) three blanks.
We'll call this pattern of strokes and blanks
the \emph{machine code} of the machine.

% The even/odd machine from section~\ref{sec:tm},
% for example,
% would be represented by
% the following pattern of strokes and blanks,
% which we'll call its \emph{Turing code}:
% % \begin{equation*}
% %   (q_{0}, 1, B, R, q_{1}), \quad
% %   (q_{0}, B, 1, R, q_{2}), \quad
% %   (q_{1}, 1, B, R, q_{0}), \quad
% %   (q_{1}, B, 1, R, q_{0}).
% % \end{equation*}
% \[
%   \inlineturingtape{{1, ,1,1, ,1, ,1,1, ,1,1, , , ,1, ,1, ,1,1, ,1,1, ,1,1,1, , , ,1,1, ,1, ,1, ,1, ,1,1, ,1, , , ,1,1, ,1, ,1,1, ,1,1, ,1}}{100}
% \]

Next,
we need to design a Turing machine $U$ that
can read the code of a machine $M$ and
simulate the behaviour of that machine for any input $I$.
The input for $U$ is
the machine code of $M$,
followed by, say, four blanks,
followed by the input $I$ for $M$.

While simulating $M$,
$U$ will divide its tape into three parts.
The left part will store the code of $M$.
The right part is a simulation of $M$'s tape.
The middle part is a working area.

\begin{quote}
  \centering
  \begin{tabular}{ccc}
    \cellcolor{gray!10} MACHINE CODE & \cellcolor{gray!10} WORK AREA & \cellcolor{gray!10} SIMULATED TAPE AREA
  \end{tabular}
\end{quote}

$U$ is going to simulate each step of running $M$ on $I$.
To this end,
$U$ needs to keep track of
$M$'s position on its tape.
We achieve this by adding a marker for the position of $M$'s head in the simulated tape area.
To make space for the marker,
we begin by inserting a blank in between any two cells of $M$'s input,
so that the original input lies in the odd-numbered squares.
For example, if the original input $I$ was
\[
   \turingtape{{1,1, ,1, ,1,1, , , , , , , }}{100}
\]
then this is converted to
\[
  \turingtape{{1, ,1, , , ,1, , , ,1, ,1, }}{100}
\]
in the simulated tape area.
The even-numbered cells can now be used to mark the position of $M$'s head.
At the beginning,
$M$'s head is positioned on the first cell of its input;
$U$ marks this by putting a stroke into the first even-numbered cell of the simulated tape area:
\[
  \turingtape{{1,1,1, , , ,1, , , ,1, ,1, }}{1}
\]

$U$ also needs to keep track of $M$'s current state.
To this end,
it simply stores the code of the state
(a single stroke for $q_0$, two strokes for $q_1$, and so on)
in the work area.
Initially,
$U$ writes a single stroke there,
assuming that every machine starts in $q_{0}$.

After this preparatory work,
the simulation begins.
It goes as follows.

\emph{Stage 1.}
Find the active position in the simulated tape area,
by moving right until you meet the first even square with a stroke.
The cell to your left holds the symbol currently scanned by $M$.
Remember this symbol by
going into distinct states depending on whether it is a blank or a stroke.

\emph{Stage 2.}
Either way, move left to the work area and print,
to the right of the code for $M$'s current state,
a blank, followed by the code of the currently scanned symbol
(1 or 11).

The machine code stored in the left part of the tape
divides into quintuple blocks,
each of which begins with a state code followed by a ``current symbol'' code.
The work area therefore now contains
the first two items of the quintuple that holds the instruction for what to do in the current state
when reading the currently scanned symbol.

\emph{Stage 3.}
Move left to find the position in the machine code that matches
the string in the work area,
preceded by three blanks.
If there's no match,
the simulation is finished.
In this case,
erase everything but the simulated tape area,
shift the content of that area to omit all the even-numbered cells,
and halt.
If there is a match,
continue to stage 4.

\emph{Stage 4.}
Scan the instructions in the matched quintuple:
remember the symbol to be written onto the tape and the direction to move
by going into a different state depending on which symbol is to be written
and in which direction to move.

\emph{Stage 5.}
Copy the last element of the quintuple
(the new state)
into the work area,
erasing the previous content of the work area.
Then move to the marked position in the simulated tape area,
write the remembered symbol
into the cell before the marker stroke.
Move the marker in the remembered direction
by two steps
(because the simulated tape contains the marker spaces).
Return to stage 1.

All this is relatively straightforward,
albeit tedious,
to implement.
The most fiddly part is stage 3,
where we need to find the position in the machine code matching the string in the work area.
This requires keeping track of positions in both strings,
which can be done by storing the positions in the work area.
If the work area runs out of space,
or the simulated machine runs off the left edge
of the simulated tape area,
a subroutine has to be called that
moves the entire content of the simulated tape area to the right.

As described,
this design is highly inefficient.
But it illustrates a profound fact:
a single mechanical architecture can in principle carry out any computation.
All modern computers are based on this insight.
You don't have to re-wire your laptop or phone whenever you want to run a new program.
Instead,
you load the program code (the algorithm) into memory
and tell the processor to read and execute that code.

\begin{exercise}
  Show that if a two-place function $f$ is Turing-computable,
  then so is the one-place function $g$ such that $g(x) = f(x,x)$.
\end{exercise}

\begin{exercise}
  Can the universal Turing machine simulate itself?
  What happens if you feed $U$ its own machine code as input,
  together with some further input $I$ for $U$?
\end{exercise}

\begin{exercise}
  According to the Church-Turing Thesis,
  any effective, mechanical algorithm can be implemented by a Turing machine.
  Use the Church-Turing Thesis to argue that
  there is a universal Turing machine.
\end{exercise}

\section{Uncomputability}\label{sec:tm-uncomputability}

In the previous chapter,
I argued that
the set of algorithms is computably enumerable,
and inferred that
there can be no algorithm that
detects whether any given algorithm halts on a given input.
We can now see how this plays out for Turing machines.

% Each Turing machine can be specified by a finite list of quintuples,
% each of which has the form $\t{q_{i}, s, s', d, q_{j}}$,
% where $i$ and $j$ are natural numbers,
% $s$ and $s'$ are '1' or 'B',
% and $d$ is 'L' or 'R'.
% It is easy to define an algorithm for enumerating all such lists of quintuples,
% and thereby all Turing machines.

% As we saw in the previous chapter,
% it follows that
% there can't be an effective algorithm that
% detects whether a given Turing machine returns any output for a given input.

Every Turing machine has a finite number of states.
A Turing machine can still run forever:
by going into an infinite loop.
Here,
for example,
is a machine that,
when started on a consecutive string of strokes,
keeps expanding that string on both ends,
without ever halting.
(A real computer would ``crash'' when running this kind of program.)

\begin{center}
  \begin{tikzpicture}[node distance=5cm, thick]
  \node[state]                 (q0) {$q_{0}$};
  \node[state, right of=q0]    (q1) {$q_{1}$};

  \path[->] (q0) edge[loop above] node {1:1R} (q0);  % march right over 1s
  \path[->] (q0) edge[bend left=20]  node[above] {B:1L} (q1);  % write 1 at right end and turn around
  \path[->] (q1) edge[loop above]  node {1:1L} (q1); % march left over 1s
  \path[->] (q1) edge[bend left=20]  node[below] {B:1R} (q0); % write 1 at left end and turn around
\end{tikzpicture}

\end{center}

From the flow chart,
it is easy to see that this machine will never halt,
no matter its input.
But is there a general recipe
for determining whether a given Turing machine will halt,
on a given input?
This is the \textit{halting problem} for Turing machines.

To be clear,
the problem is not to determine,
for a fixed machine $M$ and input $I$,
whether $M$ will halt on $I$.
\emph{This} problem is trivially decidable.
Rather,
the problem is to find a general algorithm that decides,
for any Turing machine $M$ and any input $I$,
whether $M$ halts on $I$.

\begin{exercise}
   Why is it trivially decidable whether a fixed Turing machine $M$ halts on a fixed input $I$?
\end{exercise}
% (The answer is either 'yes' or 'no',
% so it is decided by one of these two algorithms:
% (1) `write down "yes"', (2) `write down "no"'.)

We can show that the halting problem
can't be solved by a Turing machine.
A Turing machine $H$ that solves the halting problem would
take the code of a Turing machine $M$ and an input $I$ for $M$ as input
and would output (say) two strokes if $M$ halts on $I$
and one stroke if $M$ doesn't halt on $I$.
We can show by a diagonal argument that
such a machine $H$ can't exist.

\begin{theorem}{}{halting}
  The Halting Problem is undecidable by a Turing machine.
\end{theorem}

\begin{proof}
  \emph{Proof.}
  Suppose for reductio that
  there is a Turing machine $H$ that decides the Halting Problem.
  We could then plug $H$ into a larger machine $D$ that
  takes the code for a machine $M$ as input
  and halts iff $M$ halts when given \emph{its own code} (the code of $M$) as input.

  This machine $D$ would be constructed as follows.
  When started on the code of a machine $M$,
  it first creates a copy of the input.
  It then runs $H$ on the contents of the tape,
  to determine if $M$ halts on its own code.
  If the answer is yes,
  $D$ goes into an infinite loop.
  If the answer is no,
  $D$ halts.

  Now we get a contradiction if we ask whether $D$ halts on its own code:
  by design,
  $D$ halts on the code of $M$ iff
  $M$ does not halt on its own code;
  so $D$ halts on its own code iff
  $D$ does not halt on its own code.
  It follows that $D$ can't exist,
  and therefore that $H$ can't exist.
  \qed
\end{proof}

% ** From the unsolvability of the Halting Problem to incompleteness and undecidability

% We also have an uncomputable \emph{arithmetical} function,
% if we code TMs and their input as numbers.
% Of course,
% this function is ``arithmetical'' only in the sense that it takes numbers to numbers.
% It doesn't seem genuinely arithmetical in that
% its specification involves non-arithmetical concepts.
% It turns out, in fact, that
% there is a purely arithmetical definition of the halting function,
% in terms of addition and multiplication.
% But to show this,
% we have some more work to do in the next two chapters.

% ** The busy beaver problem

% \begin{exercise}
%   Explain why the set of Turing machines that halt on the empty tape is computably enumerable.
% \end{exercise}

\begin{exercise}
  Can a universal Turing machine get stuck in an infinite loop?
  If so, how?
  Could we prevent it by, say,
  keeping a counter of the number of simulated steps and abort the simulation if that counter exceeds some fixed limit?
\end{exercise}

The unsolvability of the halting problem
can be used to show that
various other functions are not Turing-computable.
A neat example is the \emph{Busy Beaver function} $\Sigma$,
introduced by Tibor Rado in 1962.
This function takes a number $n$ as input
and returns the largest number of strokes that can be printed by a Turing machine with $n$ states before halting,
when started on a blank tape.

For example,
it is easy to see that
$\Sigma(1) = 1$.
Let $M$ be any machine with just one state, $q_0$.
When started on a blank tape,
the first instruction $M$ executes is the one for $q_0$ and a blank cell.
The machine can either print a stroke or leave the cell blank;
then it moves either left or right,
to another blank cell.
If the machine doesn't halt at this point,
it will again be in state $q_0$, reading a blank cell;
it will repeat the same action,
moving in the same direction,
without end.
So the only way $M$ can halt is by halting after the first step.
The most it can print in that step is a single stroke.
So the largest number of strokes that a 1-state machine can print before halting
(when started on an empty tape) is 1.

A somewhat more involved argument along the same lines shows that $\Sigma(2) = 4$
and $\Sigma(3) = 6$.
(In exercise~\ref{ex:bb3},
I asked you to draw the flow chart for the 3-state machine that prints 6 strokes
on an empty tape and then halts.)
% It's not obvious if $\Sigma$ is defined for $0$,
% but if we allow for an ``empty'' machine with no states,
% then $\Sigma(0)$ is obviously 0:
% the empty machine can't print anything.

If the halting problem were decidable,
we could easily compute the Busy Beaver function.
For any input number $n$,
we would simply enumerate all Turing machines with $n$ states,
use the halting algorithm to discard the non-halting machines,
and run the remaining machines (on an empty tape)
to see how many strokes they print before halting.
Due to the undecidability of the halting problem,
this algorithm doesn't work.
In fact,
there is no algorithm that computes the Busy Beaver function.
More precisely,
there is no Turing machine that computes the Busy Beaver function.
This can be shown by
showing that any machine that computes the Busy Beaver function
could be used to solve the halting problem.
But it can also be shown directly:

\begin{theorem}{(Rado 1962)}{bb}
  The Busy Beaver function is not Turing-computable.
\end{theorem}
\begin{proof}
  We'll show that
  every Turing-computable total function $f$ on the natural numbers
  is eventually overtaken by the Busy Beaver function $\Sigma$.
  That is,
  for every Turing-computable total function $f$ on $\mathbb{N}$,
  there is a number $k$ such that $\Sigma(k) > f(k)$.
  If $\Sigma$ were Turing-computable,
  there would be a number $k$ such that $\Sigma(k) > \Sigma(k)$.
  This is impossible.
  So $\Sigma$ is not Turing-computable.

  Let $f$ be any Turing-computable total function from $\mathbb{N}$ to $\mathbb{N}$.
  Then the following function $g$ is also Turing-computable and total:
  \[
    g(x) = \max(f(2x), f(2x+1)) + 1.
  \]
  To compute $g(x)$ for any $x$,
  we first create a copy of the input $x$ at a sufficient distance from the original input.
  Then we use the ``times 2'' machine to convert the input $x$ into $2x$,
  and run the machine that computes $f$ on the resulting block of strokes.
  We then have $f(2x)$ on that part of the tape.
  Next, we use the ``times 2'' machine and the  ``add 1'' machine
  to convert the copy of $x$ into $2x+1$,
  and run the machine that computes $f$ on the resulting block.
  We now have $f(2x)$ and $f(2x+1)$ on the tape.
  To finish the computation of $g(x)$,
  we run your algorithm for computing $\max$ from exercise~\ref{ex:max},
  add a single stroke to the result,
  and halt.

  Let $M$ be some such machine for computing $g$.
  If $M$ has $k$ states,
  we can define, for any input $x$, a machine $N_{x}$ with $x+k$ states that
  first writes $x$ strokes on the tape and then imitates $M$.
  (No more than $x$ states are needed to write $x$ strokes.)
  
  When started on a blank tape,
  $N_{x}$ writes $g(x)$ strokes and then halts.
  So there is a machine with $x+k$ states that prints $g(x)$ strokes on the empty tape and then halts.
  By definition of the Busy Beaver function,
  this means that $\Sigma(x+k) \geq g(x)$.
  By definition of $g$,
  both $f(2x)$ and $f(2x+1)$ are less than $g(x)$.
  So we have
  \begin{gather*}
    \Sigma(x+k) \geq g(x) > f(2x); \\
    \Sigma(x+k) \geq g(x) > f(2x+1).
  \end{gather*}
  But obviously, if $x \geq k$ then
  \begin{gather*}
    \Sigma(2x+1)\geq \Sigma(2x) \geq \Sigma(x+k).
  \end{gather*}
  Combining these inequalities, we infer that $f(x)<\Sigma(x)$ for $x \geq 2k$.
  \qed
\end{proof} % From Enderton p.17

I mentioned above that the first few values of the Busy Beaver function are not hard to determine:
$\Sigma(0) = 0$, $\Sigma(1) = 1$, $\Sigma(2) = 4$, and $\Sigma(3) = 6$.
It is also known that $\Sigma(4)=13$ and $\Sigma(5)=4098$.
As of 2025,
the value of $\Sigma(6)$ is not known exactly;
but it is known that
there is a 6-state machine that prints $2 \uparrow\uparrow (2 \uparrow\uparrow (2 \uparrow\uparrow 10))$ strokes.
So $\Sigma(6)$ is at least $2 \uparrow\uparrow (2 \uparrow\uparrow (2 \uparrow\uparrow 10))$.
The up-arrow stands for repeated exponentiation:
$2 \uparrow \uparrow 10$ is $2^{2^{\cdot^{2}}}$ with ten twos in the tower.
This number is \emph{much} larger than, say, the number of atoms in the observable universe.
$2 \uparrow\uparrow (2 \uparrow\uparrow 10)$ is a power tower of $2 \uparrow\uparrow 10$ twos.
You couldn't write down all the twos in this tower even if you
managed to write a `2' onto each atom in the universe.
$2 \uparrow\uparrow (2 \uparrow\uparrow (2 \uparrow\uparrow 10))$ is a power tower of $2 \uparrow\uparrow (2 \uparrow\uparrow 10)$ twos.
$\Sigma(7)$ is known to be at least $2\uparrow^{11}(2\uparrow^{11} 3)$,
which I won't even try to explain.
It is an incomprehensibly large number.
You can inspect the machine tables for the known Busy Beaver champions
at \href{https://bbchallenge.org/~pascal.michel/bbc}{bbchallenge.org/\textasciitilde{}pascal.michel/bbc}.



% \begin{exercise}
%   Explain, informally, why every uncomputable total function from $\mathbb{N}$ to $\mathbb{N}$ must
%   at some point overtake every computable total function from $\mathbb{N}$ to $\mathbb{N}$.
% \end{exercise}

As Turing realised,
we can also use the undecidability of the halting problem
to show that
Hilbert's Entscheidungsproblem is unsolvable by a Turing machine:
there can be no Turing machine that
decides whether any given first-order sentence is valid.
The idea is that
for any Turing machine $M$ and input $I$,
we can construct a first-order sentence $S_{M,I} \to H_{M,I}$ that is valid iff
$M$ halts on input $I$.
The antecedent $S_{M,I}$ is a first-order description of the machine and its input;
the consequent $H_{M,I}$ says that the machine halts.
If we could decide whether $S_{M,I} \to H_{M,I}$ is valid
(or equivalently, whether $S_{M,I}$ entails $H_{M,I}$),
we could decide whether $M$ halts on input $I$.

To explain what $S_{M,I}$ and $H_{M,I}$ look like,
let the \emph{configuration} of a machine $M$ with input $I$ at step $n$ consist of
the machine's state, the position of its head on the tape, and the tape's content at step $n$.
$S_{M,I}$ will specify the initial configuration of $M$ on input $I$, at step 0.
It will also describe how the configuration changes from one step to the next,
in accordance with the machine table of $M$.
We'll need some non-logical vocabulary to spell this out.

I'll use `$0$' and `$s$' to create
terms for the computation steps:
`$0$' denotes step 0, `s(0)' step 1, and so on.
For the tape positions,
I use a constant `$o$' (``origin'') for the square at which the machine starts,
and two unary function symbols `$l$' and `$r$'
that move one square to the left and right, respectively.
So `$l(l(o))$', for example, denotes the square two to the left of the starting square.
I'll also introduce a predicate `$Q_{i}$' for each state $q_{i}$ of $M$,
so that $Q_{i}(n)$ means that at step $n$ the machine is in state $q_{i}$.
Finally,
I'll use two binary predicates `$@$' and `$1$',
where $@(n,x)$ means that at step $n$ the machine is positioned on square $x$,
and $1(n,x)$ that at step $n$ there is a stroke in square $x$.

With this vocabulary,
we can express the configuration of $M$ on input $I$ at every step $n$.
For example, suppose $M$ starts in state $q_{0}$ on input $11$.
Then the initial configuration can be expressed as follows.
\[
  Q_0(0) \land @(0,o) \land 1(0,o) \land 1(0,r(o)) \land \forall y(y \neq o \land y \neq r(o) \to \neg 1(0,y)).
\]

We can also express how the configuration changes from one step to the next.
For example,
if $M$ has an entry $\t{q_{0}, 1, \mathrm{B}, \mathrm{R}, q_{1}}$ in its machine table,
then $S_{M,I}$ would have the following conjunct:
\begin{multline*}
  \forall x \forall y((Q_0(x) \land @(x,y) \land 1(x,y)) \to \\
  \quad (Q_1(s(x)) \land @(s(x),r(y)) \land \neg 1(s(x),y) \land \forall z(z \neq y \to (1(s(x),z) \leftrightarrow 1(x,z))) )).
\end{multline*}
This says that
if at some step $x$ the machine is in state $q_{0}$ and positioned at some square $y$ that contains a stroke,
then at step $s(x)$ the machine is in state $q_{1}$,
positioned at the square to the right of $y$,
where the square $y$ is now blank,
and all other squares have the same content as before.

$S_{M,I}$ will be a big conjunction
containing,
first,
the initial configuration of $M$ on input $I$,
then all the transition rules for $M$,
and finally some background ``axioms''
to fix the intended interpretation of the non-logical symbols:

\begin{axioms}
  T1 & $\forall x\, s(x) \neq 0$ \\
  T2 & $\forall x \forall y (s(x) = s(y) \to x = y) $ \\
  T3 & $ \forall y (r(l(y)) = y  \land l(r(y)) = y )$ \\
  T4 & $ \forall x (Q_0(x) \lor Q_1(x) \lor \cdots \lor Q_n(x)) $ \\
  T5 & $ \forall x \forall y (Q_i(x) \to \neg Q_j(x)) \text{ for } i \neq j $ \\
  T6 & $ \forall x \exists y @(x,y) $ \\
  T7 & $ \forall x \forall y \forall z ((@(x,y) \land @(x,z)) \to y = z).$
\end{axioms}

Let's turn to $H_{M,I}$.
This is meant to say that
$M$ halts on input $I$.
It is a disjunction,
each disjunct of which corresponds to a state/symbol combination
for which there is no entry in the machine table.
For example, if there's no entry in the table for
what to do in state $q_{1}$ when reading a blank,
then $H_{M,I}$ will have the following as a disjunct:
\[
  \exists x \exists y( Q_{1}(x) \land @(x,y) \land \neg 1(x,y)).
\]
This says that
at some step $x$ the machine is in state $q_{1}$ and
positioned at a square $y$ that is blank.

\begin{theorem}{Turing-undecidability of first-order logic (Turing 1936)}{undecidability-turing}
 No Turing machine can decide whether any given first-order sentence is valid.
\end{theorem}
\begin{proof}
  \emph{Proof sketch.}
  Let $M$ be any Turing machine and $I$ any input for $M$.
  Let $S_{M,I}$ and $H_{M,I}$ be as above.
  We show that $M$ halts on $I$ iff $S_{M,I} \models H_{M,I}$.
  It follows that any Turing machine that solves the Entscheidungsproblem
  could be used to solve the halting problem.

  \emph{Left to right.}
  Suppose $M$ halts on $I$ after $n$ steps.
  Let $\Mod{M}$ be any model of $S_{M,I}$.
  One can show by induction on $n$ that
  $\Mod{M}$ satisfies the sentence describing the configuration of $M$ on input $I$ at step $n$.
  Since $M$ halts on $I$ at step $n$,
  it follows that $\Mod{M}$ satisfies $H_{M,I}$.

  \emph{Right to left.}
  Suppose $M$ does not halt on $I$.
  We can then build a model $\Mod{M}$ of $S_{M,I}$ that does not satisfy $H_{M,I}$,
  by giving all the non-logical symbols their intended interpretation. \qed
\end{proof}

I've omitted a lot of details here.
Filling them in would take a few more pages.
Since I'll give a full proof of Theorem~\ref{thm:undecidability-turing},
via a rather different route,
in Chapter~\ref{ch:incompleteness},
I'll save us the labour.

\begin{exercise}
  The proof of Theorem~\ref{thm:undecidability-turing} shows that
  $M$ halts on $I$ iff $S_{M,I} \models H_{M,I}$.
  Is it also true that $M$ \emph{doesn't} halt on $I$ iff $S_{M,I} \models \neg H_{M,I}$?
  (a) Explain why this would contradict
  the undecidability of the halting problem,
  given the completeness of first-order logic.
  (b) Explain informally how it can be that
  $M$ doesn't halt on $I$,
  but $S_{M,I} \not\entails \neg H_{M,I}$.
\end{exercise}

The proof of Theorem~\ref{thm:undecidability-turing} doesn't just show that
no Turing machine can solve the Entscheidungsproblem.
It shows more concretely that
any Turing machine that could solve the Entscheidungsproblem
could be converted into a Turing machine that solves the halting problem:
if you gave a Turing machine an ``oracle'' for deciding validity in predicate logic
-- a magical subroutine that decides validity --
then that machine could solve the halting problem.
In this sense,
the halting problem \emph{reduces to} the Entscheidungsproblem.
Many other problems have been revealed as undecidable in this way,
by showing that their solution would yield a solution to the halting problem.

\begin{exercise}
  Could the oracle Turing machine just described solve
  the halting problem for oracle Turing machines,
  or only for ordinary Turing machines?
\end{exercise}

% \begin{exercise}
% Using Church's thesis, prove Trakhtenbaum's Theorem.
% \end{exercise}

% \begin{exercise}
%   Let the \textit{power} of a formula be the size of its smallest model.
%   Consider the function that maps any number to the largest power of a formula of that length.
%   Explain why this function is not computable, given that first-order logic is undecidable.
% \end{exercise}

% \begin{exercise}
%   What's wrong with this argument against the Church-Turing thesis?
%   ``We can effectively enumerate all Turing machines in a list $M_{1}, M_{2}, M_{3}, \ldots$.
%   Define the function $d(n)$ so that $d(n) = M_{n}(n) + 1$,
%   where $M_{n}(n)$ is the output of machine $M_{n}$ on input $n$.
%   This function $d$ is computable,
%   but it can't be computed by any machine on our list:
%   if it were computed by $M_{n}$ for some $n$,
%   we'd have $M_{n}(n) = M_{n}(n) + 1$.''
%   % Point is: this diagonal machine must be undefined for its own input, so one can't add 1.
% \end{exercise}


%%% Local Variables:
%%% mode: latex
%%% TeX-master: "logic3.tex"
%%% End:
