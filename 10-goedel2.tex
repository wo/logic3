\chapter{The unprovability of consistency}

\section{Fineness of grain}

The Montagovian point...


Remember that our models are implausibly coarse-grained. There's more to the meaning of a sentence than its truth-value, or to the meaning of a predicate than its extension.

This doesn't matter if the language doesn't have operators that depend on more than extension.

We could introduce a second-order identity predicate. Then we might want more fine-grained models, so as not to validate the Fregean axiom.

Or we could introduce other non-truth-functional sentence operators. In modal
logic, we introduce a box. Models now include world-relative assignments.

How fine-grained the models should be depends on the distinctions we want to
draw. We can have operators like Prov that are sensitive to very fine-grained
distinctions.

But there are surprising limitations to what operators we can have! \ldots
