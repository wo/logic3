\chapter{The unprovability of consistency}

\section{Provability predicates}

Note that the S4 conditions are very weak.
They by no means capture provability in particular.
E.g., 'x is the gn of a sentence' also satisfies them.



PA can't prove Con(PA).
But we believe that Con(PA) is true.
So we can add it as a further axiom.
In the resulting theory,
we can prove Con(PA).
But the second incompleteness theorem will still apply:
we can't prove Con(PA+Con(PA)),
even though it's true.
We can add it as an axiom.
And so on.
We might even consider the theory defined by
the union of \emph{all} such additions,
with infinitely many ever-stronger consistency assertions.
This theory,
too,
won't prove its own consistency.
We can add its consistency as an axiom.
And so on.

But incompleteness is inescapable.


\section{Löb's Theorem}
\label{sec:lobs-theorem}

Gödel's proof of Incompleteness with his sentence `I am not provable' skirts the Liar paradox.
The proof of Löb's Theorem skirts Curry's Paradox.
Curry's Paradox goes like this.

Let $S$ be the sentence `if $S$ is true then Santa Claus exists'.
Suppose that $S$ is true.
Then it's true that if $S$ is true then Santa Claus exists
(as that is what $S$ says).
But we've assumed that $S$ is true.
By Modus Ponens,
we can infer that Santa Claus exists.
Discharging the assumption,
we've shown that \emph{if} $S$ is true then Santa Claus exists.
That's just what $S$ says.
So $S$ is true.
Since we've shown that if $S$ is true then Santa Claus exists,
we can conclude that Santa Claus exists.

Proof of Löb's Theorem. (Hamkins p.251?)


\begin{exercise}
  Is there a consistent extension of PA than can prove its own inconsistency?
\end{exercise}

\begin{exercise}
  What's the difference between the hypothesis $PA \proves A$ and the arithmetical
  hypothesis $Prov(A)$? Can one be true without the other? % cf Hamkins ex 7.9
\end{exercise}

\begin{exercise}
  Suppose someone provides a list of new arithmetical axioms from which one can prove that this list is consistent. What does this tell us? % Hamkins 7.11
\end{exercise}


\section{Fineness of grain}

The Montagovian point...


Remember that our models are implausibly coarse-grained. There's more to the meaning of a sentence than its truth-value, or to the meaning of a predicate than its extension.

This doesn't matter if the language doesn't have operators that depend on more than extension.

We could introduce a second-order identity predicate. Then we might want more fine-grained models, so as not to validate the Fregean axiom.

Or we could introduce other non-truth-functional sentence operators. In modal
logic, we introduce a box. Models now include world-relative assignments.

How fine-grained the models should be depends on the distinctions we want to
draw. We can have operators like Prov that are sensitive to very fine-grained
distinctions.

But there are surprising limitations to what operators we can have! \ldots

%%% Local Variables:
%%% mode: latex
%%% TeX-master: "logic3.tex"
%%% End:
